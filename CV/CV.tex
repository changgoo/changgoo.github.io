\documentclass[12pt,preprint,letterpaper]{aastex63}
\usepackage{mycvstyle}
\pagestyle{CV}
\begin{document}
\begin{center}
{\Large \bf Curriculum Vitae}\\
%\vspace*{0.1cm}
{Chang-Goo Kim (cgkim@astro.princeton.edu)}
\end{center}

%%========================================================================================
Department of Astrophysical Sciences 
\hfill +1-609-933-1180\\
Princeton University 
\hfill \url{http://changgoo.github.io} \\
4 Ivy Lane, Princeton 
\hfill \href{http://orcid.org/0000-0003-2896-3725}{ORCID: 0000-0003-2896-3725}\\
NJ 08544, USA
\hfill \url{cgkim@astro.princeton.edu}\\

%%========================================================================================
\itemtitle{Current Position}

\elements{Sep 2018 -- }{Associate Research Scholar}{Department of Astrophysical Sciences, Princeton University}

%%========================================================================================
\itemtitle{Employment}

\elements{Sep 2017 -- \\Aug 2018}{Flatiron Research Fellow}{Center for Computational Astrophysics (CCA), Flatiron Institute}

\elements{Sep 2016 -- \\Aug 2017}{Associate Research Scholar}{Department of Astrophysical Sciences, Princeton University}

\elements{Sep 2013 -- \\Aug 2016}{Postdoctoral Research Associate}{Department of Astrophysical Sciences, Princeton University}

\elements{Oct 2011 -- \\Aug 2013}{CITA National Fellow}{Department of Physics and Astronomy, University of Western Ontario, Canada}

\elements{Mar 2011 -- \\Aug 2011}{BK21 Postdoctoral Research Fellow}{Department of Physics and Astronomy, Seoul National University, Korea}

%%========================================================================================
\itemtitle{Education}

\elements{Mar 2005-- \\Feb 2011}{Ph.~D in Astronomy, \textnormal{Advisor: Prof. Woong-Tae Kim}}{Department of Physics and Astronomy, Seoul National University, Korea}

\elements{Mar 2001-- \\Feb 2005}{B.~S in Astronomy}{Department of Physics and Astronomy, Seoul National University, Korea}

%%========================================================================================
\itemtitle{Teaching Experience}

\elements{2019 -- \emph{present}}{Lachlan Lancaster, \textnormal{Graduate student at Princeton University}}{\textit{Globular Cluster Formation in Giant Molecular Clouds} -- Ph. D. thesis project (with Jeong-Gyu Kim and Eve Ostriker)}

\elements{2019 -- 2020}{Ryan Golant, \textnormal{Undergraduate student at Princeton University}}{\textit{Effect of early feedback in regulating star formation rates} -- Summer research, Senior thesis (with Eve Ostriker)}

\elements{2018 -- 2019}{Erin Kado-Fong, \textnormal{Graduate student at Princeton University}}{\textit{Diffuse ionized gas in star-forming galactic disks} -- Semester project (with Jeong-Gyu Kim and Eve Ostriker)}

\elements{2018 -- 2019}{Aditi Vijayan, \textnormal{Graduate student at the Indian Institute of Science}}{\textit{Kinematics and dynamics of multiphase outflows} -- Summer research via \href{https://kspa.soe.ucsc.edu/2018}{Kavli Summer Program in Astrophysics} (with Lucia Armillotta,  Eve Ostriker, Miao Li)}

\elements{2018 -- 2019}{Kareem El-Badry, \textnormal{Graduate student at the UC Berkeley}}{\textit{Evolution of supernovae-driven superbubbles with conduction and cooling} -- Summer research via \href{https://kspa.soe.ucsc.edu/2018}{Kavli Summer Program in Astrophysics} (with Eve Ostriker)}

\elements{2018}{Mohammad Refat, \textnormal{Undergraduate student at the CUNY}}{\textit{Metallicity fluctuations in TIGRESS} -- Summer research via \href{http://cunyastro.org/astrocom/}{AstroCom NYC}}

\elements{2018 -- 2019}{Erin Flowers, \textnormal{Graduate student at Princeton University}}{\textit{Turbulence driving and outflows by clustered Supernovae} -- Semester project (with Eve Ostriker)}

\elements{2017 -- \emph{present}}{Woorak Choi, \textnormal{Graduate student at Yonsei University}}{\textit{Ram pressure stripping in resolved multiphase ISM simulations} -- Ph.D thesis project (with Aeree Chung)}

\elements{2014 -- 2015 }{Roberta Raileanu, \textnormal{Undergraduate student at Princeton University}}{\textit{Superbubbles in the multiphase ISM and the loading of galactic winds} -- Junior Thesis and Summer research (with Eve Ostriker)}

\elements{2005 -- 2010 }{Teaching Assistant, \textnormal{Seoul National University}}{\footnotesize Grading problem sets and leading problem-solving sessions for courses including \emph{Solar System Astronomy and Lab., Astronomical Observation \& Lab. I \& II, Astronomy and Lab., Introduction to Astrophysics I \& II, Stars and Stellar Systems, Man \& the Universe}. \\ Designing and leading the Lab classes. \\
Teaching programming languages and analysis tools including Fortran, C, and IDL.}

%%========================================================================================
\itemtitle{Grants}

% \onelineelements{2020--2022}{PI}{NASA ATP (declined); \$409,071}
% \onelineelements{2020}{PI}{Hubble Theory Grant (declined); \$150,000}
\onelineelements{2020}{PI}{Chandra Theory Grant; \$85,000}
% \onelineelements{2019}{PI}{Chandra Theory Grant (rejected); \$85,000}
\onelineelements{2018--2021}{Co-I}{NASA TCAN (PI: Julian Borrill); \$1,398,099}

%%========================================================================================
\itemtitle{Observing Proposals}

\onelineelements{2019}{Co-I}{VLA Extra Large proposal (PI: Adam Leroy), submitted}
\onelineelements{2019}{Co-I}{VLA Regular proposal (PI: Woorak Choi), 7.4 hours, rank B}
% \onelineelements{2018}{Co-I}{ALMA cycle 6 (PI: Rommy A. Castillo), declined}

%%========================================================================================
\itemtitle{Computing Time Allocations}

% \onelineelements{2020--2022}{33M CPU hrs (1.2M SBUs), PI}{NASA Pleiades (submitted)}
% \onelineelements{2019}{60M CPU hrs, Co-I}{ASCR Leadership Computing Challenge (submitted; PI: Alex Lazarian)}
\onelineelements{2018--2021}{80M CPU hrs, Co-I}{NERSC, (PI: Julian Borrill)}
\onelineelements{2016--2019}{22M CPU hrs (800k SBUs), Co-I}{NASA Pleiades, (PI: Eve Ostriker)}

%%========================================================================================
\itemtitle{Professional Activities and Service}

\elements{2020 -- }{HI Working Group Member, \textnormal{\href{https://gaskap.anu.edu.au}{GASKAP}}}{\footnotesize
high spectral resolution survey of the HI and OH lines in the Milky Way and Magellanic Systems
}

\elements{2019 -- }{Working Group Member, \textnormal{SPICA Nearby Galaxies}}{\footnotesize 
member of the SPICA science case development team for ``Diffuse gas in galaxies''
}

\elements{2018 -- 2021}{Subnet Leader, \textnormal{NASA Theoretical and Computational Astrophysics Networks}}{\footnotesize 
leading the MHD simulation subnet in the multi-institutional collaboration funded by NASA entitled ``Modeling Polarized Galactic Foregrounds for CMB Missions''
%consists of four node institutions (Princeton, UC San Diego, UC Berkeley, and U. of Wisconsin-Madison) 
%with three subnets (MHD simulation, foreground modeling, and data synthesis).\\
%I'm leading the subnet for MHD simulations.
}

\elements{2017 -- 2022}{Working Group Leader, \textnormal{\href{https://www.simonsfoundation.org/flatiron/center-for-computational-astrophysics/galaxy-formation/smaug/}{SMAUG (Simulating Multiscale Astrophysics to Understand Galaxies)
collaboration}}}{\footnotesize 
leading the working group for ``Resolved ISM, Star formation, and Stellar feedback'' in the international collaboration funded by the Simons Foundation 
%entitled ``Simulating Multi-scale Astrophysics to Understand Galaxies''
%, consisting 9 PIs from 6 institutions (CCA, Princeton, Harvard, UC Berkeley, Zurich, Heidelberg) and $>$40 members. 
%The collaboration aims to build a fully predictive galaxy formation theory
%utilizing next-generation cosmological simulations 
%with physics-based subgrid models for small-scale baryonic physics.\\
%I'm leading the working group ``Resolved ISM, Star Formation, and Stellar Feedback.''
}

\elements{2017 -- 2019 }{Working Group Member, \textnormal{PICO collaboration}}{\footnotesize 
contributing galactic foreground modeling for a probe-class mission concept study funded by NASA entitled ``Probe of Inflation and Cosmic Origins''}

\onelineelements{2020}{Reviewer}{NASA FINESST}
\onelineelements{2017}{Review Panelist}{NSF AAG Program}
\onelineelements{2016 -- 2017}{Organizer}{Star Formation/ISM Rendezvous Seminars at Princeton University}
\onelineelements{2012 -- }{Referee}{ApJ, ApJL, MNRAS}

%%========================================================================================
\itemtitle{Invited Reviews}

\onelineelements{2019}{Invited Review}{\href{https://mist2019.sciencesconf.org/resource/page/id/2}{Cosmic turbulence and magnetic fields: physics of baryonic matter across time and scales}, Carg\'ese, France}

\onelineelements{2019}{Invited Review}{\href{https://www.aao.gov.au/conference/australia-eso-conference-2019}{Linking galaxies from the Epoch of initial star-formation to today}, Sydney, Australia}

\onelineelements{2016}{Invited Review}{\href{https://agenda.albanova.se/conferenceDisplay.py?confId=5696}{How Galaxies Form Stars}, Stockholm, Sweden}

\itemtitle{Invited Colloquia}

\onelineelements{2020}{Colloquium}{University of Georgia, Athens, GA -- remote talk}
\onelineelements{2020}{Colloquium}{University of Waterloo, Waterloo, ON, Canada}
\onelineelements{2019}{Colloquium}{University of Maryland, College Park, MD}
\onelineelements{2019}{Colloquium}{Australia National University, Canberra, Austrailia}
\onelineelements{2018}{Colloquium}{Yonsei University, Seoul, Korea}
\onelineelements{2018}{Colloquium}{Korea Astronomy and Space Science Institute, Daejeon, Korea}
\onelineelements{2017}{Colloquium}{Osaka University, Osaka, Japan}
\onelineelements{2017}{Colloquium}{University of California, Santa Barbara, CA}
\onelineelements{2016}{Colloquium}{Shanghai Jiao Tong University, Shanghai, China}
\onelineelements{2016}{Colloquium}{Korea Astronomy and Space Science Institute, Daejeon, Korea}
\onelineelements{2016}{Colloquium}{Seoul National University, Seoul, Korea}
\onelineelements{2014}{Colloquium}{Korea Astronomy and Space Science Institute, Daejeon, Korea}
\onelineelements{2014}{Colloquium}{Seoul National University, Seoul, Korea}
\onelineelements{2014}{Colloquium}{Korea Institute for Advanced Study, Seoul, Korea}
\onelineelements{2011}{Colloquium}{National Institute for Mathematical Sciences, Daejeon, Korea}
\onelineelements{2011}{Colloquium}{Yonsei University, Seoul, Korea}

\itemtitle{Conference/Workshop/Seminar}

\onelineelements{2020}{Invited Talk}{\href{https://cmb-s4.org/wiki/index.php/UChicago-2020:_Dust_Magnetic_Fields_Galaxy}{CMB-S4 Workshop}, UChicago, via Zoom}
\onelineelements{2020}{Invited Talk}{Midwest Magnetic Field Meeting 2020, Madison, WI -- cancelled due to COVID-19}
\onelineelements{2020}{Invited Talk}{\href{https://caffelattes.sciencesconf.org}{Cosmological Analyses Featuring Galactic Foreground Emission}, Lattes, France -- cancelled due to COVID-19}
% \onelineelements{2019 (planned)}{Invited Talk}{The self-organized star formation process, Orsay, France}
\onelineelements{2019}{Contributed Talk}{Feedback and its Role in Galaxy Formation, Spetses, Greece}
\onelineelements{2019}{Poster}{Linking the Milky Way and Nearby Galaxies, Helsinki, Finland}
\onelineelements{2019}{Invited Talk}{Multi-phase Gas Workshop, CCA, New York, NY}
\onelineelements{2019}{Invited Talk}{Athena++ Workshop 2019, UNLV, Las Vegas, NV}
\onelineelements{2018}{Contributed Talk}{THINKSHOP15, Potsdam, Germany}
\onelineelements{2018}{Invited Talk}{The Milky Way in the age of Gaia, Orsay, France}
\onelineelements{2018}{Invited Talk}{Kavli Summer Program in Astrophysics, CCA, New York, NY}
\onelineelements{2018}{Invited Talk}{MPPC Workshop, Princeton, NJ}
\onelineelements{2018}{Invited Talk}{CMB Foreground Workshop at CCA, New York, NY}
\onelineelements{2018}{Invited Talk}{Computational Galaxy Formation at Ringberg Castle, Germany}
\onelineelements{2017}{Invited Talk}{CMB Foreground Workshop at UCSD, San Diego, CA}
\onelineelements{2017}{Invited Talk}{The ISM beyond 3D, Orsay, France}
\onelineelements{2017}{Invited Talk}{Astrophysics Seminar, UCSB, Santa Barbara, CA}
\onelineelements{2016}{Invited Talk}{7th East-Asia Numerical Astrophysics Meeting, Beijing, China}
\onelineelements{2016}{Invited Talk}{Computational Galaxy Formation at Ringberg Castle, Germany}
\onelineelements{2015}{Contributed Talk}{Magnetic Fields in the Universe V, Carg\'ese, France}
\onelineelements{2015}{Contributed Talk}{IAU Symposium \#315, Honolulu, HI}
\onelineelements{2015}{Invited Talk}{IAS Informal Seminar, IAS, Princeton, NJ}
\onelineelements{2014}{Invited Talk}{6th East-Asia Numerical Astrophysics Meeting, Suwon, Korea}
\onelineelements{2014}{Invited Talk}{KITP Program -- Gravity's Loyal Opposition, Santa Barbara, CA}
\onelineelements{2013}{Invited Talk}{CITA National Fellow Meeting, Toronto, Canada}
\onelineelements{2013}{Contributed Talk}{KAS Spring Meeting, Daecheon, Korea}
\onelineelements{2012}{Invited Talk}{IAU General Assembly -- SpS12, Beijing, China}
\onelineelements{2012}{Contributed Talk}{AAS Meeting \#221, Long Beach, CA}

%%========================================================================================
% \itemtitle{Other information}

% \subtitle{Courses you wish to teach:} astrophysics (PH 481), computational (astro)physics (PH413), gas (fluid) dynamics, interstellar medium and star formation, classical mechanics (PH 221/222), electromagnetism (PH 231/232), introduction to computer application (CC 510)

% \subtitle{Desired research areas:} astrophysics, interstellar medium, star formation, galaxy formation, computational radiation magnetohydrodynamics

% \subtitle{Proposed starting date:} September 1, 2020

%%========================================================================================
\itemtitle{References}
\begin{itemize}
% \item \subtitle{Woong-Tae Kim} --
% \url{wkim@astro.snu.ac.kr}, +82-2-880-6769\\
% Professor, Department of Physics and Astronomy, Seoul National University 

\item \subtitle{Eve Ostriker} --
\url{eco@astro.princeton.edu}, +1-609-258-7240\\
Professor, Department of Astrophysical Sciences, Princeton University

\item \subtitle{James Stone} --
\url{jmstone@ias.edu} (\url{ddunbar@ias.edu} for letter request), +1-609-734-8054\\
Professor, School of Natural Sciences, Institute for Advanced Study 
% Emeritus Professor, Department of Astrophysical Sciences, Princeton University

\item \subtitle{Rachel Somerville} --
\url{rsomerville@flatironinstitute.org}, +1-848-445-8964\\
Group Leader, Center for Computational Astrophysics, Flatiron Institute

\item \subtitle{Greg Bryan} --
\url{gbryan@astro.columbia.edu}, +1-212-854-6837\\
Group Leader, Center for Computational Astrophysics, Flatiron Institute\\
Professor, Department of Astronomy, Columbia University

\item \subtitle{Raphael Flauger} --
\url{flauger@physics.ucsd.edu}, +1-858-534-7504\\
Professor, Department of Physics, University of California, San Diego

\item \subtitle{Snezana Stanimirovi\'c} --
\url{sstanimi@astro.wisc.edu}, +1-608-890-1458\\
Professor, Department of Astronomy, University of Wisconsin-Madison


% Distinguished Professor (on leave), Department of Physics and Astronomy, Rutgers University

\end{itemize}
% %\subtitle{David Spergel} --
% %\url{dspergel@flatironinstitute.org}, +1-609-258-3589\\
% %Director, Center for Computational Astrophysics, Flatiron Institute \\
% %Professor, Department of Astrophysical Sciences, Princeton University

% %\subtitle{Amiel Sternberg} --
% %\url{amiel@astro.tau.ac.il}, 03-6407590\\
% %Professor, Department of Astronomy, Tel Aviv University \\
% %Senior Research Scientist, Center for Computational Astrophysics, Flatiron Institute 

% Additional letters  -- 
% Woong-Tae Kim (\url{wkim@snu.ac.kr}; Seoul National University, thesis advisor),
% % %James Stone (Princeton), 
% Snezana Stanimirovi\'c (\url{sstanimi@astro.wisc.edu}; UW-Madison), 
% % Greg Bryan (CCA/Columbia),
% Rachel Somerville (\url{rsomerville@flatironinstitute.org}; CCA/Rutgers), 
% % Rapahel Flauger (\url{flauger@physics.ucsd.edu}; UCSD),
% % David Spergel (CCA/Princeton),
% % Shantanu Basu (Western),
% Amiel Sternberg (\url{amiel@astro.tau.ac.il}; Tel Aviv/MPE/CCA)

% \end{document}
\clearpage
\begin{center}
{\Large \bf Bibliography} (\href{https://ui.adsabs.harvard.edu/search/q=\%3Dauthor\%3A\%22kim\%2Cchang-goo\%22}{ADS}, 
\href{https://scholar.google.com/citations?user=jBOsJgoAAAAJ&hl=en}{Google Scholar})\\
{\student{Name}: student primarily mentored by me}\\
%\vspace{-1em}
\input{summary}\\
% \href{https://jcr.clarivate.com/JCRJournalHomeAction.action}{Impact factor (2018)} -- ApJ: 5.58 -- ApJS: 8.311 -- ApJL: 8.374 -- MNRAS: 5.231
\end{center}

%\section*{Publications --- \href{\adsurl}{{\it ADS search}}}
\itemtitle{Refereed Publications}
\begin{itemize}[itemsep=1pt]
    \input{pubs_ref}
\end{itemize}

\itemtitle{Papers under Review}
\begin{itemize}[itemsep=1pt]
    \input{pubs_arXiv}
\end{itemize}

\itemtitle{Refereed Conference Proceedings}
\begin{itemize}[itemsep=1pt]
\item[2.] \boldname{} and E.~C. {Ostriker}, 2016, In P.~{Jablonka},
  P.~{Andr{\'e}}, and F.~{van der Tak}, editors, {\em From Interstellar Clouds
  to Star-Forming Galaxies: Universal Processes?}, volume 315 of {\em IAU
  Symposium}, pages 38--41, \emph{{Feedback Regulated Turbulence, Magnetic
  Fields, and Star Formation Rates in Galactic Disks}}.
\item[1.] \boldname{}, E.~C. {Ostriker}, and W.-T. {Kim}, 2015,
  Highlights of Astronomy, 16:609--610, \emph{{Numerical modeling
  of multiphase, turbulent galactic disks with star formation feedback}}.
\end{itemize}


%\item \boldname{} and Eve Ostriker, \emph{Numerical Simulations of Multiphase
%  Winds and Fountains from Star-Forming Galactic Disks: III.
%  Characteristics of \ion{C}{4} and \ion{O}{6} Emitting Gas in TIGRESS}
%\item \boldname{}, Yuan-Sen Ting, and Mohammad Refat, \emph{Statistics of Metallicity Fluctuations in TIGRESS}
% \item \student{Woorak Choi}, \boldname{}, and Aeree Chung, \emph{Numerical Simulations of the Multiphase, Turbulence, Magnetized ISM Interacting with ICM Ram Pressure}
% % \item Kwang-Il Seon and \boldname{}, \emph{Lyman-alpha Radiation Transfer: I. the Wouthuysen-Field Effect}
% \item Bon-Chul Koo, \boldname{}, and Sangwook Park, \emph{Radiative Supernova Remnants and Supernova Feedback}

% \end{enumerate}

\end{document}
