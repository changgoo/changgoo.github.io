\documentclass[12pt,preprint,letter]{aastex63}
\usepackage{mycvstyle}
\pagestyle{CV}
\begin{document}
\begin{center}
{\Large Curriculum Vitae --
%\vspace*{0.1cm}
Chang-Goo Kim }
\end{center}

%%========================================================================================
Department of Astrophysical Sciences 
\hfill +1-609-933-1180\\
Princeton University 
\hfill \url{http://changgoo.github.io} \\
4 Ivy Lane, Princeton 
\hfill \href{http://orcid.org/0000-0003-2896-3725}{ORCID: 0000-0003-2896-3725}\\
NJ 08544, USA
\hfill \url{cgkim@astro.princeton.edu}\\

%%========================================================================================
\itemtitle{Education}

\elements{Mar 2005-- \\Feb 2011}{Ph.~D in Astronomy, \textnormal{Advisor: Prof. Woong-Tae Kim}}{Department of Physics and Astronomy, Seoul National University, Korea}
\elements{Mar 2001-- \\Feb 2005}{B.~S in Astronomy}{Department of Physics and Astronomy, Seoul National University, Korea}

%%========================================================================================
\itemtitle{Current Position}

\elements{Sep 2018 -- }{Associate Research Scholar}{Department of Astrophysical Sciences, Princeton University}

%%========================================================================================
\itemtitle{Employment}

\elements{Sep 2017 -- \\Aug 2018}{Flatiron Research Fellow}{Center for Computational Astrophysics (CCA), Flatiron Institute}
\elements{Sep 2016 -- \\Aug 2017}{Associate Research Scholar}{Department of Astrophysical Sciences, Princeton University}
\elements{Sep 2013 -- \\Aug 2016}{Postdoctoral Research Associate}{Department of Astrophysical Sciences, Princeton University}
\elements{Oct 2011 -- \\Aug 2013}{CITA National Fellow}{Department of Physics and Astronomy, University of Western Ontario, Canada}
\elements{Mar 2011 -- \\Aug 2011}{BK21 Postdoctoral Research Fellow}{Department of Physics and Astronomy, Seoul National University, Korea}


%%========================================================================================
\itemtitle{Teaching Experience}

\elements{2019 -- \emph{present}}{Sanghyuk Moon, \textnormal{Graduate student at Seoul National University}}{\textit{Star Formation in Galactic Nuclear Rings} -- Ph. D. thesis project (with Woong-Tae Kim and Eve Ostriker)}
\elements{2019 -- \emph{present}}{Lachlan Lancaster, \textnormal{Graduate student at Princeton University}}{\textit{Globular Cluster Formation in Giant Molecular Clouds} -- Ph. D. thesis project (with Jeong-Gyu Kim and Eve Ostriker)}
\elements{2017 -- \emph{present}}{Woorak Choi, \textnormal{Graduate student at Yonsei University}}{\textit{Ram pressure stripping in resolved multiphase ISM simulations} -- Ph.D thesis project (with Aeree Chung)}
\elements{2019 -- 2020}{Ryan Golant, \textnormal{Undergraduate student at Princeton University}}{\textit{Effect of early feedback in regulating star formation rates} -- Summer research, Senior thesis (with Eve Ostriker)}
\elements{2018 -- 2020}{Alwin Mao, \textnormal{Graduate student at Princeton University}}{\textit{Bound gas, Dense gas, and Star
Formation: a Deceptively Simple Braid} -- Ph. D. thesis project (with Eve Ostriker)}
\elements{2018 -- 2019}{Erin Kado-Fong, \textnormal{Graduate student at Princeton University}}{\textit{Diffuse ionized gas in star-forming galactic disks} -- Semester project (with Jeong-Gyu Kim and Eve Ostriker)}
\elements{2018 -- 2019}{Aditi Vijayan, \textnormal{Graduate student at the Indian Institute of Science}}{\textit{Kinematics and dynamics of multiphase outflows} -- Summer research via \href{https://kspa.soe.ucsc.edu/2018}{Kavli Summer Program in Astrophysics} (with Lucia Armillotta,  Eve Ostriker, Miao Li)}
\elements{2018 -- 2019}{Kareem El-Badry, \textnormal{Graduate student at the UC Berkeley}}{\textit{Evolution of supernovae-driven superbubbles with conduction and cooling} -- Summer research via \href{https://kspa.soe.ucsc.edu/2018}{Kavli Summer Program in Astrophysics} (with Eve Ostriker)}
\elements{2018}{Mohammad Refat, \textnormal{Undergraduate student at the CUNY}}{\textit{Metallicity fluctuations in TIGRESS} -- Summer research via \href{http://cunyastro.org/astrocom/}{AstroCom NYC}}
\elements{2018 -- 2019}{Erin Flowers, \textnormal{Graduate student at Princeton University}}{\textit{Turbulence driving and outflows by clustered Supernovae} -- Semester project (with Eve Ostriker)}
\elements{2014 -- 2015 }{Roberta Raileanu, \textnormal{Undergraduate student at Princeton University}}{\textit{Superbubbles in the multiphase ISM and the loading of galactic winds} -- Junior Thesis and Summer research (with Eve Ostriker)}
\elements{2005 -- 2010 }{Graduate Student Instructor (Teaching Assistant), \textnormal{Seoul National University}}{- Grading problem sets and leading problem-solving sessions for courses including \emph{Solar System Astronomy and Lab., Astronomical Observation \& Lab. I \& II, Astronomy and Lab., Introduction to Astrophysics I \& II, Stars and Stellar Systems, Man \& the Universe}. \\ - Designing and leading the Lab class for Introduction to Astronomy\\
- Teaching programming languages and analysis tools including Fortran, C, and IDL. \\
- Teaching scientific computing and numerical analysis -- root-finding, numerical integration, linear algebra, linear regression }

%%========================================================================================
\itemtitle{Grants}

\onelineelements{2020}{PI}{Chandra Theory Grant; \$85,000}
\onelineelements{2018--2021}{Co-I}{NASA TCAN (PI: Julian Borrill); \$1,398,099}

%%========================================================================================
\itemtitle{Observing Proposals}

\onelineelements{2019}{Co-I}{VLA Extra Large proposal (PI: Adam Leroy), under review}
\onelineelements{2019}{Co-I}{VLA Regular proposal (PI: Woorak Choi), 7.4 hours, rank B}

%%========================================================================================
\itemtitle{Computing Time Allocations}

\onelineelements{2018--2021}{80M CPU hrs, Co-I}{NERSC, (PI: Julian Borrill)}
\onelineelements{2016--2019}{22M CPU hrs (800k SBUs), Co-I}{NASA Pleiades, (PI: Eve Ostriker)}

%%========================================================================================
\itemtitle{Professional Activities and Service}

\elements{2017 -- 2022}{Working Group Leader, \textnormal{\href{https://www.simonsfoundation.org/flatiron/center-for-computational-astrophysics/galaxy-formation/smaug/}{SMAUG (Simulating Multiscale Astrophysics to Understand Galaxies)
collaboration}}}{\footnotesize 
leading the working group for ``Resolved ISM, Star formation, and Stellar feedback'' in the international collaboration funded by the Simons Foundation 
}
\elements{2018 -- 2021}{Subnet Leader, \textnormal{NASA Theoretical and Computational Astrophysics Networks}}{\footnotesize 
leading the MHD simulation subnet in the multi-institutional collaboration funded by NASA entitled ``Modeling Polarized Galactic Foregrounds for CMB Missions''
}
\elements{2020 -- }{HI Working Group Member, \textnormal{\href{https://gaskap.anu.edu.au}{GASKAP}}}{\footnotesize
high spectral resolution survey of the HI and OH lines in the Milky Way and Magellanic Systems
}
\elements{2019 -- }{Working Group Member, \textnormal{SPICA Nearby Galaxies}}{\footnotesize 
member of the SPICA science case development team for ``Diffuse gas in galaxies''
}
\elements{2017 -- 2019 }{Working Group Member, \textnormal{PICO collaboration}}{\footnotesize 
contributing galactic foreground modeling for a probe-class mission concept study funded by NASA entitled ``Probe of Inflation and Cosmic Origins''}
\onelineelements{2020}{Reviewer}{NASA FINESST}
\onelineelements{2017}{Review Panelist}{NSF AAG Program}
\onelineelements{2016 -- 2017}{Organizer}{Star Formation/ISM Rendezvous Seminars at Princeton University}
\onelineelements{2012 -- }{Referee}{ApJ, ApJL, MNRAS}

%%========================================================================================
\itemtitle{Invited Reviews}

\onelineelements{2019}{Invited Review}{\href{https://mist2019.sciencesconf.org/resource/page/id/2}{Cosmic turbulence and magnetic fields: physics of baryonic matter across time and scales}, Carg\'ese, France}
\onelineelements{2019}{Invited Review}{\href{https://www.aao.gov.au/conference/australia-eso-conference-2019}{Linking galaxies from the Epoch of initial star-formation to today}, Sydney, Australia}
\onelineelements{2016}{Invited Review}{\href{https://agenda.albanova.se/conferenceDisplay.py?confId=5696}{How Galaxies Form Stars}, Stockholm, Sweden}

\itemtitle{Invited Colloquia}

\onelineelements{2020}{Colloquium}{University of Georgia, Athens, GA -- remote talk}
\onelineelements{2020}{Colloquium}{University of Waterloo, Waterloo, ON, Canada}
\onelineelements{2019}{Colloquium}{University of Maryland, College Park, MD}
\onelineelements{2019}{Colloquium}{Australia National University, Canberra, Australia}
\onelineelements{2018}{Colloquium}{Yonsei University, Seoul, Korea}
\onelineelements{2018}{Colloquium}{Korea Astronomy and Space Science Institute, Daejeon, Korea}
\onelineelements{2017}{Colloquium}{Osaka University, Osaka, Japan}
\onelineelements{2017}{Colloquium}{University of California, Santa Barbara, CA}
\onelineelements{2016}{Colloquium}{Shanghai Jiao Tong University, Shanghai, China}
\onelineelements{2016}{Colloquium}{Korea Astronomy and Space Science Institute, Daejeon, Korea}
\onelineelements{2016}{Colloquium}{Seoul National University, Seoul, Korea}
\onelineelements{2014}{Colloquium}{Korea Astronomy and Space Science Institute, Daejeon, Korea}
\onelineelements{2014}{Colloquium}{Seoul National University, Seoul, Korea}
\onelineelements{2014}{Colloquium}{Korea Institute for Advanced Study, Seoul, Korea}
\onelineelements{2011}{Colloquium}{National Institute for Mathematical Sciences, Daejeon, Korea}
\onelineelements{2011}{Colloquium}{Yonsei University, Seoul, Korea}

\itemtitle{Conference/Workshop/Seminar}

\onelineelements{2020}{Invited Talk}{\href{https://cmb-s4.org/wiki/index.php/UChicago-2020:_Dust_Magnetic_Fields_Galaxy}{CMB-S4 Workshop}, UChicago -- remote talk}
\onelineelements{2020}{Invited Talk}{Midwest Magnetic Field Meeting 2020, Madison, WI -- cancelled due to the pandemic}
\onelineelements{2020}{Invited Talk}{\href{https://caffelattes.sciencesconf.org}{Cosmological Analyses Featuring Galactic Foreground Emission}, Lattes, France -- cancelled due to the pandemic}
\onelineelements{2019}{Contributed Talk}{Feedback and its Role in Galaxy Formation, Spetses, Greece}
\onelineelements{2019}{Poster}{Linking the Milky Way and Nearby Galaxies, Helsinki, Finland}
\onelineelements{2019}{Invited Talk}{Multi-phase Gas Workshop, CCA, New York, NY}
\onelineelements{2019}{Invited Talk}{Athena++ Workshop 2019, UNLV, Las Vegas, NV}
\onelineelements{2018}{Contributed Talk}{THINKSHOP15, Potsdam, Germany}
\onelineelements{2018}{Invited Talk}{The Milky Way in the age of Gaia, Orsay, France}
\onelineelements{2018}{Invited Talk}{Kavli Summer Program in Astrophysics, CCA, New York, NY}
\onelineelements{2018}{Invited Talk}{MPPC Workshop, Princeton, NJ}
\onelineelements{2018}{Invited Talk}{CMB Foreground Workshop at CCA, New York, NY}
\onelineelements{2018}{Invited Talk}{Computational Galaxy Formation at Ringberg Castle, Germany}
\onelineelements{2017}{Invited Talk}{CMB Foreground Workshop at UCSD, San Diego, CA}
\onelineelements{2017}{Invited Talk}{The ISM beyond 3D, Orsay, France}
\onelineelements{2017}{Invited Talk}{Astrophysics Seminar, UCSB, Santa Barbara, CA}
\onelineelements{2016}{Invited Talk}{7th East-Asia Numerical Astrophysics Meeting, Beijing, China}
\onelineelements{2016}{Invited Talk}{Computational Galaxy Formation at Ringberg Castle, Germany}
\onelineelements{2015}{Contributed Talk}{Magnetic Fields in the Universe V, Carg\'ese, France}
\onelineelements{2015}{Contributed Talk}{IAU Symposium \#315, Honolulu, HI}
\onelineelements{2015}{Invited Talk}{IAS Informal Seminar, IAS, Princeton, NJ}
\onelineelements{2014}{Invited Talk}{6th East-Asia Numerical Astrophysics Meeting, Suwon, Korea}
\onelineelements{2014}{Invited Talk}{KITP Program -- Gravity's Loyal Opposition, Santa Barbara, CA}
\onelineelements{2013}{Invited Talk}{CITA National Fellow Meeting, Toronto, Canada}
\onelineelements{2013}{Contributed Talk}{KAS Spring Meeting, Daecheon, Korea}
\onelineelements{2012}{Invited Talk}{IAU General Assembly -- SpS12, Beijing, China}
\onelineelements{2012}{Contributed Talk}{AAS Meeting \#221, Long Beach, CA}

%%========================================================================================
\itemtitle{References}

\begin{itemize}%[itemsep=1pt,topsep=\parskip]
% \item \subtitle{Woong-Tae Kim} --
% \url{wkim@astro.snu.ac.kr}, +82-2-880-6769\\
% Professor, Department of Physics and Astronomy, Seoul National University 

\item \subtitle{Eve C. Ostriker} \\
\url{eco@astro.princeton.edu}, +1-609-258-7240\\
Professor, Department of Astrophysical Sciences, Princeton University

\item \subtitle{Rachel S. Somerville} (co-sign with Prof. Bryan) \\
\url{rsomerville@flatironinstitute.org}, +1-848-445-8964\\
Group Leader, Center for Computational Astrophysics, Flatiron Institute

\item \subtitle{Greg L. Bryan} (co-sign with Prof. Somerville) \\
\url{gbryan@astro.columbia.edu}, +1-212-854-6837\\
Group Leader, Center for Computational Astrophysics, Flatiron Institute\\
Professor, Department of Astronomy, Columbia University

\item \subtitle{James M. Stone} \\
\url{jmstone@ias.edu}, +1-609-734-8054\\
Professor, School of Natural Sciences, Institute for Advanced Study 
% Emeritus Professor, Department of Astrophysical Sciences, Princeton University

% \item \subtitle{Amiel Sternberg} (independent testimonial) \\
% \url{amiel@astro.tau.ac.il}, 03-6407590\\
% Professor, Department of Astronomy, Tel Aviv University \\
% Senior Research Scientist, Center for Computational Astrophysics, Flatiron Institute

\item \subtitle{Raphael Flauger} (cosmological implications)\\
\url{flauger@physics.ucsd.edu}, +1-858-534-7504\\
Professor, Department of Physics, University of California, San Diego

\clearpage

\begin{center}
{\bf List of Publications} (\href{https://ui.adsabs.harvard.edu/search/q=\%3Dauthor\%3A\%22kim\%2Cchang-goo\%22}{ADS}, 
\href{https://scholar.google.com/citations?user=jBOsJgoAAAAJ&hl=en}{Google Scholar})\\
{\student{Name}: student advised/co-advised by me}\\
    \input{summary.tex}
\end{center}

\itemtitle{Refereed Publications \input{summary_1st.tex}}
\begin{itemize}[itemsep=1pt]
%\item \textbf{Five most significant contributions (\#20, 23, 26, 27, 29 marked by **):} I am the first and corresponding author of these papers. I led the projects, implemented numerical schemes, carried out the simulations on supercomputers in Canada (Sharcnet), Princeton (Tiger/Perseus), Flatiron Institute (Rusty), NASA (Pleiades), and NERSC (Cori),
%developed the analysis methods, created the plots and visualizations, and wrote the texts. This statement is true for all first-authored papers listed below except \#16, 17, and 18 for which Prof. Woong-Tae Kim was a corresponding author as the thesis supervisor. All co-authors contributed to the analysis and interpretation of the results, and the manuscript writing.
\input{pubs_ref_1st.tex}
\end{itemize}

\itemtitle{Refereed Publications (second author)}
\begin{itemize}[itemsep=1pt]
\input{pubs_ref_2nd.tex}
\end{itemize}

\itemtitle{Refereed Publications (co-author)}
\begin{itemize}[itemsep=1pt]
\input{pubs_ref_co.tex}
\end{itemize}

\itemtitle{Papers under Review}
\begin{itemize}[itemsep=1pt]

%\item Clark, S.~E.; \textbf{Kim, Chang-Goo}; Hill, J. Colin; Hensley, Brandon S. \textit{$TB$, or not $TB$? The origin of parity-odd quantities in the Galactic polarized dust emission}, to be submitted

%\item Pandya, V.~\textit{et al.} (incl. {\bf CGK}; 8/13) \textit{Characterizing mass, momentum, energy and metal outflow rates of multi-phase galactic winds in the FIRE-2 cosmological simulations}, submitted to MNRAS (03/2021)

%\item \student{Moon, Sanghyuk}; Kim, Woong-Tae; \textbf{Kim, Chang-Goo}; Ostriker, Eve C., \textit{Star Formation in Nuclear Rings with the TIGRESS Framework}, submitted to ApJ (02/2021)

%\item \student{Lancaster, Lachlan}; Ostriker, Eve C.; Kim, Jeong-Gyu; \textbf{Kim, Chang-Goo}, \textit{Efficiently Cooled Stellar Wind Bubbles in Turbulent Clouds II. Validation of Theory with Hydrodynamic Simulations}, submitted to ApJ (12/2020)

%\item \student{Lancaster, Lachlan}; Ostriker, Eve C.; Kim, Jeong-Gyu; \textbf{Kim, Chang-Goo}, \textit{Efficiently Cooled Stellar Wind Bubbles in Turbulent Clouds I. Fractal Theory and Application to Star-Forming Clouds}, submitted to ApJ (12/2020)

%\item Motwani, Bhawna; Genel, Shy; Bryan, Greg L.; \textbf{Kim, Chang-Goo}~\textit{et al.}, \textit{First results from SMAUG: Insights into star formation conditions from spatially-resolved ISM properties in TNG50}, 2020 (\arxiv{2006.16314}) [\href{http://adsabs.harvard.edu/abs/2020arXiv200616314M}{5 citations}], submitted to ApJ (10/2020)

\input{pubs_arxiv.tex}
\end{itemize}

\itemtitle{Conference Proceedings}
\begin{itemize}[itemsep=1pt]
\item \textbf{Kim, Chang-Goo}; Ostriker, Eve C., 2016 (\arxiv{1511.00018}), In P.~{Jablonka},
  P.~{Andr{\'e}}, and F.~{van der Tak}, editors, {\em From Interstellar Clouds
  to Star-Forming Galaxies: Universal Processes?}, volume 315 of {\em IAU
  Symposium}, pages 38--41, \doiform{10.1017/S1743921316007225}{Feedback Regulated Turbulence, Magnetic
  Fields, and Star Formation Rates in Galactic Disks}.
\item \textbf{Kim, Chang-Goo}; Ostriker, Eve C.; Kim, Woong-Tae, 2015 (\arxiv{1211.5161}),
  Highlights of Astronomy, 16:609--610, March 2015, \doiform{10.1017/S174392131401240X}{Numerical modeling
  of multiphase, turbulent galactic disks with star formation feedback}.
\end{itemize}

% \item \subtitle{Snezana Stanimirovi\'c} --
% \url{sstanimi@astro.wisc.edu}, +1-608-890-1458\\
% Professor, Department of Astronomy, University of Wisconsin-Madison

\end{itemize}


\end{document}
